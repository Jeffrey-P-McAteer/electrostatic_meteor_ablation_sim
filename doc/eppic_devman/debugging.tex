
\chapter{Debugging}

\section{Introduction}


\section{GDB}


\section{Totalview}\index{Totalview}

Totalview is propietary debugger resently installed on both the
Bluegene and Linux (Katana) BU clusters. If you need to run the
debugger in batch mode (submitting it to a batch queue), then the
following sections will discuss how to do this. Batch mode, for
example, is the only way to run programs on the Bluegene machine. On
the linux cluster, debugging of short runs can be performed
interactively, like so:
\begin{lstlisting}[language=Bash]
  # two processor run using Totalview debugger:
  mpirun -np 2 totalview eppic input.in
\end{lstlisting}

While the following sections use the Bluegene enviroment to explain
how to run Totalview using a batch-submitted job, the process is the
same for any other submission program: test that remote programs can
be displayed on your machine using the
\textbf{\$DISPLAY}\index{\$DISPLAY} environment 
variable; run 'totalview eppic' as the program; and pass the the
display setting using the '-v \$DISPLAY' option. 

\subsection{Setting your display}
While totalview will run on the Bluegene machine, it's display is sent
to your machine. \emph{On levi or lee}, the \textbf{\$DISPLAY}
environment variable needs to be set to
\textbf{$<$YOUR-IP-ADDRESS$>$:0}. This tells Bluegene where to send
the images, but you also need to set your machine to accept the
images. The following todo list assumes you are using a default Ubuntu
installation, but the instrcutions should be similar under other Linux
environments.\\ 

\noindent Do the following in a terminal: 
\begin{enumerate}
\item As root/sudo, edit /etc/X11/xinit/xserverrc:\\
  Remove the "-nolisten tcp"
\item As root/sudo, edit /etc/gdm/gdm.conf:\\
  Comment out the "DisallowTCP=true" so that it doesn't add the
  -nolisten back.  
\item Restart your display by logging out and back into your machine.
\item Add \textbf{lee.bu.edu} and \textbf{levi.bu.edu} to the list of
  allowed machines: 
\begin{lstlisting}[language=Bash]
  xhost +levi.bu.edu +lee.bu.edu
\end{lstlisting}
\item Run ifconfig, get your IP address from the value of ``inet addr:''
\item SSH into lee or levi \emph{without(!)} display forwarding, ie
  without ``-Y'' or ``-X'' flags.  
\item Set the DISPLAY environment variable to
  \textbf{$<$YOUR-IP-ADDRESS$>$:0}, this is usually done like so:

\begin{lstlisting}[language=Bash]
  # let's say your IP address is 000.000.000.000
  setenv DISPLAY 000.000.000.000:0
\end{lstlisting}
\item Test that everything is setup correctly by running
  \textbf{xclock\&}. If nothing happens,  
  you'll get a message in a minute that a connection is not setup correctly.
  Otherwise, you should see a little clock pop up. 
\end{enumerate}

\subsection{Submission Script}
Now you should be all set to submit a totalview job. To do this, you
first need to make sure eppic is configured for debbuging, by running:
\begin{lstlisting}[language=Bash]
  ./confugure --enable-debugging
\end{lstlisting}
\noindent and recompiling the code (run \textbf{make clean} before
compiling). Now take a submission script that you've used for a run
that fails; we will make two small modifications to allow Totalview to
execute EPPIC. Change the following in the script:
\begin{enumerate}
\item Change the ``\# @ executable'' line from 
\begin{lstlisting}[language=Bash]
# @ executable = /bgl/BlueLight/ppcfloor/bglsys/bin/mpirun
\end{lstlisting}
to
\begin{lstlisting}[language=Bash]
# @ executable = /usr/bin/xterm
\end{lstlisting}
\item Next, add ``-display $<$YOUR-IP-ADDRESS$>$:0 -e totalview
  mpirun -a'' to the begining of the ``\# @ arguments ='' value, like
  so, for example:
\begin{lstlisting}[language=Bash]
# @ arguments =  -display 128.197.73.25:0 -e totalview mpirun -a \
-verbose 1 -np $(numprocs) -cwd $(working_path) \
-exe $(exepath)/eppic -args "$(working_path)/$(jobname).i"
\end{lstlisting}
\item Make sure you are running in co-processor mode (-mode CO, or the
  default), otherwise Totalview will fail partially through the
  initialization. 
\end{enumerate}

Now submit this job, and as soon as it runs, you will get an xterm
window popping up on your screen, and this will be followed by two
other windows that are the Totalview main windows.

\section{Debug-Logging}