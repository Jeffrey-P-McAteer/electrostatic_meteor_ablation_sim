

\chapter{Testing}\label{c:testing}

\section{Introduction}

In this section we present our approach to testing EPPIC and its inidividual
routines. There are three aspects of EPPIC that we test: (1) the
program as a whole, (2) individual routines and modules
(\emph{e.g. the ArrayND class}) and (3) the programs output and
analysis routines (\emph{e.g. IDL processing routines}). The next
three sections discuss how to setup these tests.

\section{Eppic Testing}

This is still a work in progress... more to come soon. 

\section{Routine Testing}

Routine testing may seem like overkill, and one needs to carefully
consider what should be tested. There are two main reasons for
building these tests. First, a suprising amount of time is spent on
adding new code, and this approach provides a systematic framework for
adding code efficiently. Next, these tests provide an educational
resource for both you, the developer, and students new to programming
EPPIC or coding in general. These tests provide numours examples of
simple exercises demonstrating an isolated component found in
EPPIC. Finally, a minor purpose is to identify compatibility issues
when installing EPPIC on new machines. 
\newline
\noindent Routine testing follows a general pattern: 

\begin{itemize}
\item You develope a small c++ program 
  (\emph{subroutine}\_test.cc) that calls the routine with inputs
that will produce a predictable outcome. The outcome is tested and the
program returns 1 if it passes, 0 if it fails. 
\item Next you write a shell script (\emph{subroutine}\_test.sh) to run
  this program several times, at 
  least once for serial execution and once for parallel
  execution. This is needed because most of the routines depend on MPI
  and various senarios should be tested.
\item Then you add the shell script and the test program to the
  tests/routines/ directory and specific
  Makefile.am variable definitions, which are then used to
  appropriately generate the 'make check' target. 
\item Finally, run 'make check' until the test programs succeed. 
\end{itemize}

Once you've perfected this procedure on one machine, running 'make
check' on other machines will help identify compadability issues and
perhaps unknown bugs. The trick to making these tests useful is to
determine the right set of ideal inputs for the routines. 

\subsection{Building the test program}

As more examples are added to the code, a specific recipie will
develope. For now, writing these test programs are a bit ad-hoc. The
file \textbf{tests/routines/pic\_test.cc\_template} should be used as a
starting-off point, so copy this if your writing code to test a
PIC-related routine. As we add tests for the other routines in EPPIC
(including the initialization routines required in the above
template), more examples and templates will become
available. Unfortunately EPPIC has many global variables, and this
requires most of the test routines to declare and define nearly all of
them. This too will improve in the future as we try to rein these
in. 

In the template, you will see a \textbf{for}-loop stepping through the
number of timesteps, and this is intended to test scaling, efficencly
and memory leaks (c.f. subsection \ref{ss:testing_scaling}). Whether
you are testing these other aspects of the code or not, put your test
and call to your subroutine inside this \textbf{for}-loop, because by
default this loop runs at least once. If your test passes, do nothing
and the program should finish successfully. If your test fails,
increment the variable \textbf{failed}, like so:
\begin{lstlisting}[language=C++]
  /* test_value and real_value are made-up for this example */
  if (test_value!=real_value) failed++;
\end{lstlisting}
Finally, you should note that c++
treats return values in a funny way. 0 and 1 are acceptable, otherwise 
special system variables are needed to specify a different error
code. 

\subsection{Building the test shell script}

For now, the shell scripts that call these test programs are roughly
the same. This motivates an automated process for dealing with the
test programs, which is something we can consider in the near
future. Below we show the shell script
\textbf{efield\_quasineut\_test.sh} which runs 
\textbf{efield\_quasineut\_test.cc} and tests the
\textbf{efield\_quasineut} routine. 
% REMOVED FROM REPOSITORY SO THIS LINE WONT COMPILE CORRECTLY:
%\lstinputlisting[caption=tests/routines/efield\_quasineut\_test.sh,numbers=left,stepnumber=2]{../tests/routines/efield_quasineut_test.sh}
Starting at line 2, a small message documents what the test does. At
line 7, we print a little message saying what is being tested. Using 7
.'s ('....... ') presents a consistent message, needed when the uses
runs 'make check'. On line 9, the script sources a file customized by
the configure script, because mpi execution sometimes varies from
system to system, the routine run\_mpi (used on line 14) needs to be
customized for each system. Lines 12 and 14 show how the .cc programs
are called and tested. If they fail, the shell script exits
imideiately with a failed return value. Otherwise on the line 15 the
script ends succesfully. You can essentially copy this script for your
own tests. 

\subsection{Adding test to the Makefile}

Once you have a test program and a shell script to run it under
multiple senarios, you need to add these files to their appropriate
positions in the the automake template file \textbf{Makefile.am}. Here
we describe in detail the meaning of each piece of the file, but below
you'll find a todo list for adding your tests. First we start with
compiling the tests. As discussed elsewhere (Chapter
\ref{c:bulid_sys}), automake uses variable names to piece together
their use. Here we list the variables used in compiling the test
programs: 
%\lstinputlisting[caption=Compiling tests in
%tests/routines/Makefile.am,numbers=left,stepnumber=2,firstline=24]{../tests/routines/Makefile.am} 
The \textbf{check\_PROGRAMS} variable (line 26) lists the programs to
be compiled when the user runs 'make check'. Each \emph{program}
listed should have a corresponding \textbf{\emph{program}\_SOURCES}
variable listing the source needed to compile the program. While the
configure script will appropriately define the compiler's options, you
can customize this process with the \textbf{INCLUDES} (header file
search paths), \textbf{LDADD} (libraries and library search path), and
\textbf{DEPENDENCIES} (compilation dependencies that are not actually
compiled) variables. The \textbf{EXTRA\_DIST} variables lists files
that may get overlooked my automake during the definition of the list
of files packaged into a 'tar' file with the 'make dist' call. The
\textbf{check\_SCRIPTS} lists shell scripts that are created at some
point during the build process and should be monitored (for possible
removal and updating).  

Before we write up the todo list of things you need to do to add your
program to this file, we will describe where to list the shell script.  
%\lstinputlisting[caption=Shell scripts and listing tests in
%tests/routines/Makefile.am,numbers=left,stepnumber=2,firstline=12,lastline=23]{../tests/routines/Makefile.am}    
The variable \textbf{TESTS} lists all the tests to do when 'make
check' is run. We assign it the value of two variables, whose names
are not interpreted by automake but resemble the naming convention
used elsewhere. \textbf{TESTS\_ccprograms} holds a space-separated
list of tests that involve only a single execution of a C++
program. Since most of the tests involve multiple executions, this
variable is likely to remain empty. The more important variable
\textbf{TESTS\_shellscripts} (lines 16-19) lists the shell scripts make
should run. For clarity, each script is listed on it's own line.

Now we summarize with a small todo list:
\begin{enumerate}
  \item Add the name of your test C++ program to the
    \textbf{check\_PROGRAMS} variable. 
  \item Make a corresponiding \textbf{\emph{myprogram}\_SOURCES}
    variable that lists all the source code needed to compile your
    program \emph{myprogram}. This is usually only one file. 
  \item If needed, adjust the variables used in compiling:
    \textbf{INCLUDES} (header file 
    search paths), \textbf{LDADD} (libraries and library search path), and
    \textbf{DEPENDENCIES}.
  \item Add your shell script to the list of test shell scripts
    defined in the variable \textbf{TESTS\_shellscripts}.
\end{enumerate}

To enact your changes, return to the packages root directory. Now run
\textbf{automake} to convert the \textbf{Makefile.am} into a
\textbf{Makefile.in}. Rerun the configure script to make the
\textbf{Makefile}: \textbf{./configure}. And finally, run \textbf{make
  check}. The make file should report if the tests, including the new
one you just added, successfully ran. 
\begin{lstlisting}[caption=A typical result of running 'make check',numbers=left,stepnumber=2]
mpic++  -O3   -o efield_quasineut_test efield_quasineut_test.o 
../../src/libeppic.a ...A LOT MORE STUFF...

make[3]: `run_mpi.tests' is up to date.
make[3]: Leaving directory `/home/yannpaul/src/eppic/tests/routines'
/usr/bin/make  check-TESTS
make[3]: Entering directory `/home/yannpaul/src/eppic/tests/routines'
....... TESTING efield_test.sh
....... Single processor
....... Multi processor
PASS: efield_test.sh
....... TESTING pic_current_test.sh
....... Multi processor
....... Single processor
PASS: pic_current_test.sh
....... TESTING efield_quasineut_test.sh
....... Single processor
; WARNING: N0 for fluid adjusted!
	n0avgd0 = 1e+11
....... Multi processor
; WARNING: N0 for fluid adjusted!
	n0avgd0 = 1e+11
PASS: efield_quasineut_test.sh
....... TESTING efield_petsc_test.sh
....... Single processor
....... Multi processor
PASS: efield_petsc_test.sh
==================
All 4 tests passed
==================
make[3]: Leaving directory `/home/yannpaul/src/eppic/tests/routines'
make[2]: Leaving directory `/home/yannpaul/src/eppic/tests/routines'
make[2]: Entering directory `/home/yannpaul/src/eppic/tests'
make[2]: Nothing to be done for `check-am'.
make[2]: Leaving directory `/home/yannpaul/src/eppic/tests'
make[1]: Leaving directory `/home/yannpaul/src/eppic/tests'
\end{lstlisting}
When you run make check, you'll probably see lines similar to those on
lines 1-3, which I've abriviated here. Then on line 6 make is running
the \textbf{check-TESTS} target. Starting on line 8, each test should
run 4 lines, listing what it's doing and if it succeeds. Finally, line
29 summarizes. If you don't get to this point, let us know, because
this process should be the painless part of testing (where as getting
your C++ tests to actually succeed is usually the bottle-neck).

\subsection{Testing parallel scaling}\label{ss:testing_scaling}
In the future, we would like to make parallel scaling testing
automated.

\subsection{Testing optional code}
In the tests/routines Makefile.am file, there are some PETSC variables
that get definied only if PETSC is available. If you are using a
non-crucial-to-EPPIC library, but one that is needed for  your code,
you may want 
to include it using similar protective measures. This will allow other
people to use EPPIC even if they don't have the same libraries you
have access too. 

%\lstinputlisting[caption=Including optional code in
%tests/routines/Makefile.am,numbers=left,stepnumber=2,firstline=2,lastline=10]{../tests/routines/Makefile.am}     

The \textbf{HAVE\_PETSC} variable used in the conditional on line 2 is
actually defined in the configure script, which actaully tests if you
have access to the PETSc library. This is done with the
\textbf{AM\_CONDITIONAL M4} routine, which defines a
\emph{conditional} variable for use by \emph{A}uto\emph{M}ake
depending on the outcome of a test. Here it tests if the environment
variable \textbf{\$PETSC\_DIR} is \emph{not} empty:
\begin{lstlisting}[language=bash,title=Defining HAVE\_PETSC in the
  configure script]
AM_CONDITIONAL([HAVE_PETSC],[test ! x$PETSC_DIR = x])
\end{lstlisting}
Within the \textbf{if}-block of the Makefile.am file, variables that list the
C++ programs and test shell scripts are defined. These variables are
included in the appropriate test variables. If PETSc is not available,
then this block is not run and these library-dependant variables will
remain empty and the corresponding programs are not included. 

\section{Analysis Testing}

This section will describe testing of output routines and post EPPIC
workup. To come later. 