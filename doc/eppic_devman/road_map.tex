

\chapter{Development Road Map}
% some key counters
\newcounter{eppic_version}
\newcounter{eppic_revision}
\newcounter{eppic_minor_revision}

% set the current version of the code
\setcounter{eppic_version}{1}
\setcounter{eppic_revision}{0}
\setcounter{eppic_minor_revision}{0}

\section{Future Milestones}

\newenvironment{versionList}[1][\value{eppic_version}]{
  \begin{list}
    {eppic-\arabic{eppic_version}.\arabic{eppic_revision}}
    {\usecounter{eppic_revision}}
    \setcounter{eppic_version}{#1}
    \addtocounter{eppic_revision}{-1}
  }
  {\addtocounter{eppic_version}{1}\end{list}}
\newenvironment{minor_versionList}{
  \begin{list}
    {eppic-\arabic{eppic_version}.\arabic{eppic_revision}.\arabic{eppic_minor_revision}}
    {\usecounter{eppic_minor_revision}}
    \addtocounter{eppic_minor_revision}{-1}
  }
  {\end{list}}


\subsection{Clean-up and Prep}
In this version we are basically cleaning up any glitches with using
the autotools approach. Along with fixing some bugs, the framework for
this style of cooperative development is laid. This includes the
current documentation using LaTeX, the make target 'check' and the
affiliated test programs, a series of example inputs that would be
useful for showing how EPPIC can be used, and finally all the IDL
routines needed to process EPPIC's output. 

\begin{versionList}
% Version 1: Clean-up and Prep
\item Documentation
\item Testing
\item Examples
\item IDL and I/O
  \begin{minor_versionList}
  \item Add common routines to idl directory
  \item Add ppic3d routines to idl directory
  \item Add eppic routines to idl directory
  \end{minor_versionList}
\end{versionList}


\subsection{AF Research I}
This represents the first version of major development. It follows the
guidelines laid out in the 2008 Air Force proposal for Spread-F
research. 
\begin{versionList}
% Version 2: AF Research I
\item Particle Injection
  \begin{minor_versionList}
    \item Add test programs for the current inject facilities.
    \item Add examples for the current inject facilities. 
    \item Add global injection capabilities (including tests,etc.)      
  \end{minor_versionList}
\item Non-periodic A-la Marcos Research
  \begin{minor_versionList}
    \item Domain decompose current non-periodic code
    \item Extend to 3D with 2D spectral solve for y and z
    \end{minor_versionList}
\end{versionList}


\subsection{AF Research II}
This is a continuation of the previous version, but it is given it's
own version number here as is done in the proposal.
\begin{versionList}
% Version 3: AF Research II
\item Thermal Fluid Simulations  
\item Coulomb Collisions
\item 3D Quasi-neutral + Fluid Simulations
\end{versionList}


\section{Past Milestones}



\section{Possible Additions, Improvements and Modifications}

\subsection{Global Changes}
\begin{itemize}
\item Have a 'distribution' structure, which includes method-independent parameters, such as qd,md,nd0 etc.
\item Make a 'vecfield' class
\item Add Doxygen for documenting code.
\end{itemize}

\subsection{Fluid Methods}
\begin{itemize}
\item Add initialization method that runs fluid simulation until 'equilibrated'
\end{itemize}
\subsection{PIC Methods}
% \begin{itemize}
% \item
% \end{itemize}

\section{Known Bugs}

\subsection{General}

\subsection{Fluid Methods}

\subsection{PIC Methods}

\subsection{Field Solver}