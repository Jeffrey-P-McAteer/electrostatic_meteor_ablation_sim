
\chapter{Analysis}

Here we discuss the tools provided for analyzing the output of eppic
in the context of a few examples.  

\section{IDL}\index{IDL}

EPPIC analysis has traditionally relied on IDL for it's analysis, so
many IDL routines have been written to read in and process the output
of EPPIC. These are located in the 'idl' directory, under several
subdirectories. Make sure to add the absolute path of this diectory to
your IDL \textbf{!path} variable using the \textbf{expand\_path}
routine. 

There are some common routines for reading in arrays written by EPPIC
or for plotting and moving-making of data. Then there are several
runtime scripts that call these routines, doing anything from small
tasks, such as reading in all the density arrays (\textbf{denep.pro}),
to much larger and multiple-stepped tasks. The first section covers a
very brief overview of using IDL. This is followed by a series of
sections listing the most common routines, with examples of the
output. 

\subsection{Using IDL}
The best source for help is the program \textbf{idlhelp}, which is
really just a link to a set of HTML help pages. Typically you will run
idl in a runs root directory, which is the directory that holds all
the domain**** directories. First, you might want to calculate the
cyclotron frequency:
\begin{lstlisting}[language=ITTIDL,numbers=left,stepnumber=2]
  IDL> ; Comments start with ';'
  IDL> @eppic.i
  IDL> help,Bz
  IDL> help,qd0,md0
  IDL> print,'Omega_e = ',Bz*qd0/md0,', Hz'
\end{lstlisting}
Line 2 loads the input file, which has a syntax acceptable to
IDL. This will load a series of variables, whose value and existance
can be checked with the \textbf{help} routine (lines 2 and 3). 

... more to come on this front... not necissary now

\subsection{Saving Data, Images, and Movies}\label{ss:savingdata}

\subsection{The 'ep' Scripts}

\subsection{plot}

\subsection{image\_plot}

\subsection{image\_movie}

\section{VTK and Paraview}

For the 3D runs, VTK and Paraview provide an alternative method for
plotting the data in several different forms. VTK (visualization tool
kit) is a library for plotting data, and Paraview is a program that
provides a simple(ish) graphical interface to the VTK tools. To get
the data in VTK format, we use IDL (\emph{see Section
  \ref{ss:savingdata}}). Once in VTK format, the file (or files, if
there is time-dependant data) may be read into Paraview... more to
come on this subject.

\subsection{Basics of Paraview}






