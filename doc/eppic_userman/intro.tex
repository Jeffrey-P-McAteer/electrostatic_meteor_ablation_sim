

\chapter{Introduction}

Something snappy going here...

\chapter{Subversion}
Apache Subversion is a version control software that allows us to maintain a repository of file versions that we check in and out for editing. In order to check files out (i.e. create a local copy that you can edit), you will need to enter the 
following command at the prompt
\begin{center} svn checkout svn+ssh://katana.bu.edu/project/eregion/.astrosvn/eppic/trunk/ ./[destination_directory] \end{center}
The terminal will print a capitol A next to each item that svn adds to your local directory.

When you are ready to submit your changes - this is called committing, or checking in - make sure you are in the ``/project/eregion/[YourID]/eppic/'' directory and enter the command
\begin{center} svn commit [path] -m ``[informational message]'' \end{center}
(NB: The reason you must be in the top-level directory of the distribution you originally checked out is that that directory contains the hidden subdirectory ``.svn''. This subdirectory, in turn, contains information on your local working copy of the distribution). 

In order to make sure you are working on the latest copy of some file, you must update your working copy. This is simply done by entering the command
\begin{center} svn update \end{center}
while in the directory containing '.svn'. The terminal will print a capitol U next to each item it updates.