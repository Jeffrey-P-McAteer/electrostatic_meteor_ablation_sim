

\chapter{An EPPIC Tutorial}

\section{High frequency Electron plasma waves}
%% Example 1 -- true mass e-, just e-, what modes

As a simple introduction, example (1) studies a single distriubtion plasma. 
This is possible because the peroidic field solver is only valid for net
neutral systems, and a time-independant, uniform ion distribution is 
numerically included in the simulation of electrons. The electrons are treated
with their true mass, which is often much smaller in other simulations. This
small mass results in very vast plasma frequencies, which then requires very
fine time resolution. As we'll see in later simulaitons, we are often 
interested longer time scales, and in those cases we'll experiment with 
an artificially large electron mass. 

The goal of example one is to introduce the basics of running eppic and 
post-processing the output. This run will require at least 8 processors, but
it is setup for 64 processors. If you are running on a machine with
fewer processors, then increase the particle number in the input
accordingly. So if you only have 8 processors, increase the
\textbf{npd0} variable from \textbf{820000} to \textbf{6560000}. You
may run into memory issues by using fewer processors and more
particles per processor, so keep this in mind if you run into errors. 

What does the output look like?


%% Example 2 -- true mass e-, just e-, heating experiment


%% Example 3 -- modified mass e-, ions + e-, ion modes


%% Example 4 -- FBI simulation